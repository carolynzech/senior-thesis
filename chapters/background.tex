\chapter{Background}

\syslang{} expresses policies over \sys{}'s PDG.
%
First, we provide some background on PDGs in general, then on \sys{}'s design.

\section{PDGs}
A program dependence graph (PDG)~\cite{ferranteProgramDependenceGraph1987} represents a program's \emph{dependencies}.
%
The nodes are, generally speaking, any variables in the program, including function arguments and return values.
%
The edges represent dependencies between nodes.
%
If the PDG contains an edge $x \xrightarrow{} y$, that indicates that $y$ depends on $x$.
%
There are two kinds of dependencies: data dependencies and control dependencies.
%
A data dependency $x \xrightarrow{\mathsf{data}} y$ indicates that $x$ directly affects $y$, e.g., $y = x$, 
because data flows from $x$ into $y$.
%
A control dependency $x \xrightarrow{\mathsf{control}} y$ indicates that $x$ indirectly affects $y$, 
e.g., \lstinline[language=Rust]|if (x == 1) y = 1 else y = 2|.
%
We say that $x$ has \emph{transitive} influence on $y$ if there exists some path from $x$ to $y$ through the PDG,
either through a direct edge from $x$ to $y$ or through intermediate nodes 
(e.g., $x \xrightarrow{\mathsf{data}} z \xrightarrow{\mathsf{control}} y$).
%
$x$ has transitive \emph{data} influence on $y$ if there is a path from $x$ to $y$ comprised entirely of data edges.
%
$x$ has transitive \emph{control} influence on $y$ if there is a path from $x$ to $y$ comprised entirely of control edges.

\section{\sys{}}

\sys{} generates PDGs for Rust applications.
%
Users express their polices as assertions over their application's PDG.
%
For example, an assertion may state that $x$ must have transitive data influence on $y$.
%
\sys{} provides a Rust API, called the \emph{Graph Query API}, for common query patterns.
%
For instance, \writers{} use the \lstinline[language=Rust]|flows_to(x, y,| \lstinline[language=Rust]|EdgeSelection)| primitive to query for transitive 
\break{} influence, providing |EdgeSelection::Both|, |EdgeSelection::Data|, 
\break{} or |EdgeSelection::Control| for transitive influence,
transitive data influence, or transitive control influence, respectively.

If a \ce{} wanted to write a policy about some privacy-critical operation
(e.g., data sinks), they could manually inspect the PDG to identify all relevant nodes, 
but that would be tedious and error-prone.
%
Instead, they use \sys's \emph{markers} abstraction to annotate their source code with custom labels.
%
When \sys{} constructs the application's PDG, it attaches markers to the appropriate nodes.
%
\Devs{} apply markers to source code through inline annotations, e.g.:
\begin{lstlisting}[language=Rust]
    #[paralegal::marker(sink, argument = recipient)]
    fn send_email(recipient, sender, content) {... }
\end{lstlisting}
which applies a marker \lstinline[language=CNL]|@@sink@@| to the |recipient| argument of the |send_email| function.
%
The PDG will contain one |recipient| node for each invocation of |send_email|, with \lstinline[language=CNL]|@@sink@@| applied to each of them.

\Writers{} use \sys{}'s marker abstraction to write their \policies{}.
%
In this example, rather than explicitly enumerate the dependencies for each |recipient| node,
the \writer{} can instead specify that \emph{all} nodes marked \lstinline[language=CNL]|@@sink@@| have some dependency.
%
Markers make policies clearer because they allow \writers{} to reason about concepts like 
\lstinline[language=CNL]|@@sink@@| or \lstinline[language=CNL]|@@user data@@|,
rather than source-code level entities like function arguments.
%

After writing a policy and applying its markers to source code,
the \dev{} leverages their application knowledge to mark certain functions 
as \emph{entrypoints} for \sys{}'s analysis.
%
\sys{} constructs PDGs for each \controller{}.
%
\Controller{}s are, generally speaking, the ``privacy relevant'' functions in the application.
%
In a web application, for example, they are the user-facing endpoints where data enters and exits the application.
%
The \dev{} then runs \sys{} against their application, which generates the application's \emph{marked PDGs},
i.e., its PDGs with its markers applied to the appropriate graph nodes.
%
\sys{} checks the policy against the marked PDG and outputs whether the application is compliant.