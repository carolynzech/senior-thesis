\chapter{Evaluation}

We evaluate \syslang{} against seven popular, third-party Rust applications to answer four questions:
%
\begin{enumerate}[nosep]
    \item Can \syslang's grammar express real-world \policies? (\S\ref{sec:expressivity})
    \item What is the developer effort required to encode policies in \syslang? (\S\ref{sec:accessibility})
    \item Does \syslang's compiler correctly translate \policies to graph queries without loss of precision? (\S\ref{sec:precision})
    \item Does \syslang's compiler produce efficient, optimized queries? (\S\ref{sec:efficiency})
\end{enumerate}
%

\section{Expressivity}
\label{sec:expressivity}

For each of the applications, we encoded their \syslang{} policies by
emulating the role of a \ce{}.
%
We did not look at source code, but rather used demo applications and documentation.
%
We found that \syslang{} was expressive enough to represent all the policies we wanted to check.
%
We also found that applications could often use identical policies.
%
For instance, two applications (mCaptcha and Plume) use the same data deletion policy.
%
Three others (Hyperswitch, Lemmy, and mCaptcha) use the same access control policy
except for application-specific markers.
%
In cases where the policy was inherently dynamic, we defined static approximations.
%
For example, a GDPR data deletion policy would state that some \controller{} deletes \emph{all} of a user's data.
%
\sys{} cannot verify that the application actually deletes all of the user's data,
since the exact contents of that data is only known at runtime.
%
However, it can ensure that for each type marked |user_data|, 
there is some data of that type that goes to a |deleter|.
%
This policy is expressive enough to find bugs where applications forget to delete a given type of user data,
but cannot catch bugs where an application only deletes \emph{some} of the data of a given type.
%
\todo{table summarizing applications and policies}

\section{Precision}
\label{sec:precision}
To evaluate \syslang's precision, we wrote policies for each of the seven applications using the Graph Query API and compare them to the compiled \syslang{} policies.
%
The Rust policies use API operations (namely, reasoning about direct dependencies of a given node) that are out of \syslang's scope.
%
We ran both sets of policies on compliant versions of the applications and on versions with injected bugs.
%
We found that both versions produced identical results.
%
This result demonstrates that \syslang's worse precision is acceptable for practical privacy policies.

\section{Efficiency}
\label{sec:efficiency}
We timed each of the Graph Query API policies against their equivalent \syslang{} policies and compared the results.
%
We averaged the results over 10 runs each.
%
We would expect \syslang{} policies to be slower on average because of its compiler limitations (\S\ref{sec:limitations}).
%
We found that \syslang{} policies are 2-10\% slower than their Graph Query API counterparts,
amounting to 67-91500 milliseconds of additional time.
%
This result demonstrates that \syslang{} policies incur an acceptable overhead compared to native Rust policies.
