\chapter{Implementation}
\label{sec:limitations}
We implement the \syslang{} compiler in 2370 lines of Rust.

\section{Direct Dependencies}
\syslang{} intentionally abstracts away details of the PDG to make policies easier to write.
%
For example, its relations reason about \emph{transitive} dependencies between PDG nodes.
%
There is no way for \ces{} to enforce a \emph{direct} edge between two nodes (i.e., no intermediate data or control flow).
%
In most cases, transitive dependencies are the correct level of abstraction.
%
For example, given the code \lstinline[language=Rust]|sink(*sensitive)|,
one would reasonably conclude that |sensitive| goes to |sink|.
%
However, since |sensitive| is dereferenced, the PDG would contain an intermediate node between |sensitive| and |sink|,
so there is no direct data flow edge between |sensitive| and |sink|.\carolyn{verify w justus}
%
However, it is possible that a \dev{} would want to forbid any intermediate data transformations
to secret data before it reaches some |sink| (perhaps in a security-critical setting).
%
\syslang{} could not express this policy, while \sys{}'s Graph Query API can.

\section{Related Nodes}
Similarly, \syslang{} cannot reason about the direct children, siblings, or antecedents of a given node.
%
For example, take the function |send_email(recipient, sender, content)|.
%
A \ce{} may want to express the idea that if data marked \lstinline[language=CNL]|@@sensitive@@| flows to |content|,
then data marked \lstinline[language=CNL]|@@admin@@| flows to |recipient|.
%
However, if the \ce{} simply checked that data marked \lstinline[language=CNL]|@@admin@@| flows to \emph{some} |recipient|,
then code like this would pass the policy:
\begin{lstlisting}[language=Rust]
    send_email(admin@cs.brown.edu, student@cs.brown.edu, benign_data);
    send_email(otherstudent@cs.brown.edu, student@cs.brown.edu, sensitive_data)
\end{lstlisting}
The second |send_email| call is unsafe because sensitive data is sent to a student, not an administrator.
%
However, because the first, unrelated |send_email| recipient is an administrator, \sys{} would see that some recipient is an admin and the policy would pass.
%
To prevent this bug, a \ce{} could make their policy more precise, and instead say that if if data marked \lstinline[language=CNL]|@@sensitive@@| flows to |content|,
then data marked \lstinline[language=CNL]|@@admin@@| flows to |recipient| \emph{at the same call site}.
%
\syslang{} cannot express this policy exactly.
%
Its ``goes to ... operation'' relation can check that data marked \lstinline[language=CNL]|@@admin@@| flows to the same call site, but it cannot differentiate between arguments of the call site as the policy requires.
%
\sys{}'s Graph Query API can express the policy.

\section{Code Optimization}
The \syslang{} compiler does not optimize its outputted code, so it misses opportunities for efficiency improvements.
%
For example, if a policy contains two iterators over the same data, the \syslang{} compiled policy will traverse the PDG twice to gather the relevant nodes,
when it could have just done one traversal and stored the result for the subsequent iteration.
%
Our \syslang{} prototype supports five levels of bullets, although more could be supported with additional engineering effort.

% iterators w arbitrary conditions