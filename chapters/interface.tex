\chapter{Design}
\label{sec:interface}

A \syslang{} policy has three sections--its body, its scope, and (optionally) its definitions:
%
\begin{itemize}[nosep]
  \item The body contains the predicates that define the policy.
  \item The scope determines which analysis entry points must satisfy the body.
  \item The definitions allow \ces{} to make their bodies simpler and more efficient.
\end{itemize}

\syslang{}'s full grammar is in \Cref{sec:grammar}.

\section{Scope}
\label{sec:scope}

A \ce{} has three options for the scope of their policy:
%
\begin{enumerate}[nosep]
    \item \emph{Everywhere} indicates that every \controller{} must satisfy the body.
    \item \emph{Somewhere} indicates that at least one \controller{} must satisfy the body.
    \item \emph{In} |name| indicates that the \controller{} with name |name| must satisfy the body.
\end{enumerate}

The appropriate scope is policy-dependent.
%
For instance, if an application should always encrypt sensitive data before storage,
the \ce{} should select an ``Everywhere'' scope.
%
For a GDPR data deletion policy, however, an ``Everywhere'' scope would not make sense, 
since that would require every \controller{} storing user data to also delete it.
%
Instead, such a policy could use a ``Somewhere'' scope to mandate that at least one \controller{} deletes a user's data.
%
\Ces{} may wish to be more specific and ensure that a particular \controller{} obeys a policy,
in which case they should use the \emph{In} |name| scope.
%
This scope achieves greater precision, but is also tied to the source code; 
if the name of the \controller{} changes, the policy must change as well.
%

\section{Body}
\label{sec:body}

The policy body has three components: quantifiers, predicates, and conjunctions/disjunctions of them.
%
\subsection{Quantifiers}
Quantifiers allow \ces{} to iterate over a collection of marked items, 
reasoning about one item at a time.
%
\syslang{} provides two quantifiers: |For each| and |There is|.
%
There are four ways to use these quantifiers, depending on whether the variable is defined or not:
%
\begin{enumerate}[label=(\roman*),nosep]
  \item \lstinline[language=CNL]|For each "x" marked @@m@@:|
  \item \lstinline[language=CNL]|There is a "x" marked @@m@@ where:|
  \item \lstinline[language=CNL]|For each "x":|
  \item \lstinline[language=CNL]|There is a "x" where:|
\end{enumerate}

(i) and (ii) iterate over all items marked \lstinline[language=CNL]|@@m@@|,
while (iii) and (iv) iterate over a \emph{defined} collection of items.
%
Definitions allow \ces{} to gather a collection of items ahead of time,
filtering on a more complicated set of conditions than markers alone (\Cref{sec:definitions}).

Both types of quantifiers iterate over a collection of objects,
evaluating the body in the context of the current object \lstinline[language=CNL]|"x"|.
%
A |For each| loop succeeds if the body is true for all \lstinline[language=CNL]|"x"|.
%
A |There is| loop succeeds if the body is true for at least one \lstinline[language=CNL]|"x"|.

\paragraph{Vacuity: }
If there are no \lstinline[language=CNL]|"x"|s, the body of a |For each| loop is vacuously true.
%
\syslang{} allows for vacuity for |For each| loops because it evaluates policies on a per-\controller{} basis,
and some \controller{}s may not have certain markers.
%
For example, take the (abbreviated) Freedit policy from \S\ref{sec:overview}:
\begin{lstlisting}[language=CNL]
Scope: Everywhere

Policy:
For each "view" marked @@views@@:
    [...]
\end{lstlisting}

If |For each| quantifiers enforced non-vacuity,
this policy would fail on every \controller{} that does not 
handle \lstinline[language=CNL]|@@views@@|.
%
To avoid these failures, the \ce{} would have to use the \emph{In} |name| scope to include
only the \controller{}s that should contain \lstinline[language=CNL]|@@views@@|.
%
However, this solution would defeat \syslang's goal of being independent from source code.
%
\syslang{} allows |For each| loops to be vacuous so that \ce{}s 
can use the ``Everywhere'' scope over the \emph{In} |name| scope by default.

If a \ce{} wants to a write a stricter policy with a vacuity check,
they can leverage the |There is| quantifier,
as in Figure~\ref{f:vacuity}.
%
They should then change their scope to use the \emph{In} |name| scope,
so that \sys{} only evaluates the policy against the \controller{}s
that should contain \lstinline[language=CNL]|@@views@@|.

\begin{figure}
  \begin{lstlisting}[language=CNL]
    Policy:
    /*There is a*/ "view" /*marked*/ @@views@@ /*where:*/
      /*There is a*/ "database store" /*marked*/ @@store@@ /*where:*/
        "view" /*goes to*/ "database store"
    and
    For each "view" marked @@views@@:
      [...]
  \end{lstlisting}
  \caption{
    The Freedit policy from \S\ref{sec:overview} with a vacuity check
    that enforces that at least one \lstinline[language=CNL]|"view"| is stored.
    }
  \label{f:vacuity}
\end{figure}

\paragraph{Variables:}
Observe that the loop bodies reason about the quantifier variables.
%
If instead, \syslang{} eschewed quantifiers and wrote this policy as:
\begin{lstlisting}[language=CNL]
If a "view" marked @@views@@ goes to a "database store" marked @@store@@:
  There is a "date" marked @@time@@ where:
    "date" goes to a "database store" marked @@store@@
\end{lstlisting}
\syslang{} would not enforce that \lstinline[language=CNL]|"view"| 
and \lstinline[language=CNL]|"date"| 
go to the \emph{same} \lstinline[language=CNL]|"database store"|,
since the second instance of \lstinline[language=CNL]|"database store"| 
is again defined over all locations marked \lstinline[language=CNL]|@@store@@|.
%
Quantifiers allow for the correct version of the policy by allowing \ces{} to refer to the same object multiple times.

\subsection{Predicates}
%
Predicates are between two objects: either two variables and a variable and a marker.
%
The full list of available predicates is in Figure~\ref{f:predicates}.
%
\syslang{} also supports the negation of each of these predicates, e.g. \lstinline[language=CNL]|"a" does not go to "b"|.

\begin{figure}[h]
  \small
  \begin{tabular}{|p{6cm}|p{8cm}|}
      \hline
      \syslang{} Predicate                                                       &  Obligation on PDG                   \\ \hline
      \lstinline[language=CNL]|"a" influences "b"|                              & \lstinline[language=CNL]|"a"| has transitive influence on \lstinline[language=CNL]|"b"|. \\
      \hline
      \lstinline[language=CNL]|"a" goes to "b"|                                 &  \lstinline[language=CNL]|"a"| has transitive data flow influence on \lstinline[language=CNL]|"b"|. \\
      \hline
      \lstinline[language=CNL]|"a" affects whether "b" happens|                 & \lstinline[language=CNL]|"a"| has transitive control flow influence on \lstinline[language=CNL]|"b"|. \\
      
      \hline
      \lstinline[language=CNL]|"a" goes to "b" only via "c"|                    & On every path from \lstinline[language=CNL]|"a"| to \lstinline[language=CNL]|"b"|,
                                                                                  \lstinline[language=CNL]|"a"| passes through \lstinline[language=CNL]|"c"|. \\
      \hline
      \lstinline[language=CNL]|"a" goes to "b"'s operation|                     & \lstinline[language=CNL]|"a"| goes to the call site associated with \lstinline[language=CNL]|"b"|
                                                                                 (e.g., if \lstinline[language=CNL]|"b"| is an argument to a call site, then \lstinline[language=CNL]|"a"| goes to any of that call site's arguments). \\
      \hline 
      \lstinline[language=CNL]|"a" is marked @@m@@|                             & \lstinline[language=CNL]|"a"| is marked \lstinline[language=CNL]|@@m@@|. \\
             
    \hline
  \end{tabular}
    \caption{\syslang's predicates and the obligations they enforce on \sys's marked PDG.}
    \label{f:predicates}
\end{figure}

\subsection{Numbered Clauses} 

In our examples thus far, we have used indentation to nest quantifiers.
%
However, such a design is error-prone--with just one accidental indentation, 
a \ce{} would write an entirely different policy than what they intended.
%
For instance, take the policies in Figure~\ref{f:indentation}, which differ only in indentation but have different meanings.
%
Policy~\ref{f:indentation}(a) will fail if there is no \lstinline[language=CNL]|"x"|.
%
Policy~\ref{f:indentation}(b), however, can still pass if \lstinline[language=CNL]|"y"| goes to \lstinline[language=CNL]|"z"|.

Rather than allow a stray indent to change the meaning of the policy,
\syslang{} instead enforces that \ces{} explicitly specify the scope of each statement.
%
They do so with \emph{numbered clauses}, inspired by similar notation in legal text.
%
Policies~\ref{f:indentation}(c) and Policies~\ref{f:indentation}(d) 
are equivalent to Policy~\ref{f:indentation}(a) and Policy~\ref{f:indentation}(b),
respectively, but they use \syslang{} numbered clauses to make the scope of each statement explicit.

A numbered clause is followed by an quantifier or a predicate.
%
To introduce an additional numbered clause at the same level, 
\ces{} use |and| or |or| operators,
which indicate an unambiguous connection between the clauses.
%
\Ces{} are not permitted to mix operators (|and|s and |or|s) on the same numbered clause level,
since the operator precedence in such cases would be ambiguous.

\begin{figure}[H]
  \begin{subfigure}[b]{0.5\columnwidth}
    \begin{subfigure}[b]{\textwidth}
    \begin{lstlisting}[language=CNL]
      There is a "x" where:
        "x" goes to "y"
        or
        "y" goes to "z"
    \end{lstlisting}
    \caption{}
    \end{subfigure}
    \begin{subfigure}[b]{\textwidth}
    \begin{lstlisting}[language=CNL]
      There is a "x" where:
        "x" goes to "y"
      or
      "y" goes to "z"
    \end{lstlisting}
    \caption{}
    \end{subfigure}
  \end{subfigure}
  \begin{subfigure}[b]{0.5\columnwidth}
    \begin{subfigure}[b]{\textwidth}
    \begin{lstlisting}[language=CNL]
      1. There is a "x" where:
        A. "x" goes to "y"
        or
        B. "y" goes to "z"
    \end{lstlisting}
    \caption{}
    \end{subfigure}
    \begin{subfigure}[b]{\textwidth}
    \begin{lstlisting}[language=CNL]
      1. There is a "x" where:
        A. "x" goes to "y"
      or
      2. "y" goes to "z"
      \end{lstlisting}
      \caption{}
    \end{subfigure}
    \end{subfigure}
    \caption{Policies with identical syntax but different scopes.
    (a) and (b) use indentation to indicate scope, 
    while (c) and (d) make the scope explicit through numbered clauses.
    (These policies are partial; we elide quantifiers for brevity.)}
\label{f:indentation}
\end{figure}


\subsection{Other Details}

\paragraph{Conditionals: }

Privacy policies often impose \emph{conditional} obligations, e.g.,
\emph{if} an application stores sensitive data, \emph{then} it must obtain a user's consent.
%
\syslang{} supports conditionals through its \lstinline[language=CNL]|If p, then: q| construct,
where |p| is a predicate and |q| is either a predicate, quantifier loop, or another conditional.
%
\Ce{}s leverage \syslang{}'s numbered clauses to specify when |q| begins and ends.
%
For example, in Figure~\ref{f:conditional}, the antecedent \lstinline[language=CNL]|"x" goes to "y"| begins with numbered clause indicator |a.|.
%
Thus, the consequent |q| is every statement between |then:| and the next statement on the same ``clause level'' as the antecedent,
which is indicated with a |b.|.

\begin{figure}[h]
  \begin{lstlisting}[language=CNL]
    1. For each "x":
      A. For each "y":
        /*a. If*/ "x" goes to "y", /*then:*/
          /*i) For each*/ "z":
            /*A)*/ "z" /*does not go to*/ "y"
            /*and*/
            /*B) [...]*/
          /*or*/
          /*ii) [...]*/
        b. [...]
    \end{lstlisting}
\caption{
  A \syslang{} policy that uses a conditional (bolded).
}
\label{f:conditional}
\end{figure}


\paragraph{Types:}
\syslang{} intentionally abstracts away details of the PDG to make policies easier to write.
%
Namely, \syslang{} avoids reasoning about particular source code entities (e.g., arguments, functions).
%
The one exception to this rule is datatypes.
%
To understand why this exception is necessary, consider the following data deletion policy:
\begin{lstlisting}[language=CNL]
1. For each "sensitive" type marked @@user_data@@:
  A. There is a "source" that produces "sensitive" where:
    a. There is a "deleter" marked @@deletes@@ where:
        i) "source" goes to "deleter"
\end{lstlisting}
%
This policy enforces that for each user data type, some data of that type is deleted.
%
Consider how a \ce{} would write this policy without the |type| keyword.
%
They could write that \lstinline[language=CNL]|There is a "source" marked @@user_data@@| that is deleted,
but only checks that at least one \lstinline[language=CNL]|@@user_data@@| type is deleted, not that all of them are.
%
They could approximate the |type| keyword by giving each type a unique marker and using \lstinline[language=CNL]|There is|
quantifiers for each of them, but such a policy would be tedious to write and easy to get wrong.

\section{Definitions}
\label{sec:definitions}

In the Freedit example from \S\ref{sec:overview}, the \ce{}'s final policy was:
\begin{lstlisting}[language=CNL]
1. For each "view" marked @@views@@:
  A. For each "database store" marked @@store@@:
    a. If "view" goes to "database store" then:
      i) There is a "date" marked @@time@@ where:
        A) "date" goes to "database store"
\end{lstlisting}
This policy contains five levels of nesting.
%
As policies get more complex, many levels of nesting can make policies harder to understand.
%
To address this issue, \syslang{} allows \ces{} to create \emph{definitions}.
%
A definition declares a variable ahead of time which refers to all items that meet a certain condition.
%
Observe that the Freedit policy does not enforce any obligations on a \break{}
\lstinline[language=CNL]|"database store"| unless it stores a \lstinline[language=CNL]|"view"|.
%
Rather than iterate through \emph{all} database stores, 
a \writer{} can collect only the relevant database stores up front,
then write their policy in terms of those.
%
In this case, the Freedit \ce{} would create the following definition:
\begin{lstlisting}[language=CNL]
1. "view store" is each "store" marked @@db_store@@ where:
  A. There is a "view" marked @@views@@ where:
    a. "view" goes to "store"
\end{lstlisting}
and revise their policy to:
\begin{lstlisting}[language=CNL]
1. For each "view store":
  A. There is a "date" marked time where:
    a. "date" goes to "view store" 
\end{lstlisting}
%
Note that the latter policy is also more efficient because it avoids the double |For each| loop of the original.
%
While we do not expect efficiency to be a \ce's primary concern,
the definitions feature may be a useful way to improve performance
if \devs{} find the original policy to be too slow in practice.

Since \sys{} constructs per-\controller{} PDGs,
\syslang{} policies, once compiled, are evaluated against one \controller{} at a time.
%
The scope of the policy dictates which \controller{}(s) must uphold the policy.
%
Since policy bodies are \controller{}-specific,
definitions are \controller{}-specific by default as well.

However, a \ce{} may want to gather all marked items meeting a certain condition \emph{across \controller{}s}.
%
For instance, consider a policy that states that for each |sensitive| type that the application |store|s,
that type is also |delete|d.
%
It is unlikely that a single \controller{} would both store the sensitive data \emph{and} delete it.
%
Instead, the \ce{} could declare a definition to gather the sensitive types that are stored across the application,
then write a policy that states that some \controller{} must delete those types.
%
They do so by appending ``anywhere in the application'' to their definition declaration (Figure~\ref{f:anywhere-in-app}).

\begin{figure}[H]
\begin{lstlisting}[language=CNL]
Scope: Somewhere

Definitions:
1. "stored sensitive" is each "sensitive" type marked @@sensitive@@ where, 
/*anywhere in the application:*/
  A. There is "database store" marked @@store@@ where:
    a. "sensitive" goes to "database store"

Policy:
1. For each "stored sensitive":
  A. There is a "deleter" marked @@deletes@@ where:
    a. "stored sensitive" goes to "deleter"
\end{lstlisting}
\caption{A data deletion policy which uses a definition to gather
all sensitive types from each of the application's \controller{}s.}
\label{f:anywhere-in-app}
\end{figure}
