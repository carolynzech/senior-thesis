\section{Surface Language}
\label{sec:interface}

A \syslang{} policy has two to three sections: its scope, its definitions (optional), and its body.

\subsection{Scope}
\label{sec:scope}

A \ce{} has three options for the scope of their policy:
%
\begin{enumerate}[nosep]
    \item \emph{Everywhere} indicates that every \controller{} should obey the policy.
    \item \emph{Somewhere} indicates that at least one \controller{} should obey the policy.
    \item \emph{In} |name| indicates that the \controller{} with name |name| should obey the policy.
\end{enumerate}

The appropriate scope is policy-dependent.
%
For instance, if an application should always encrypt sensitive data before storage,
the \ce{} should select an ``Everywhere'' scope.
%
For a GDPR data deletion policy, however, an ``Everywhere'' scope would not make sense, 
since that would require every \controller{} storing user data to also delete it.
%
Instead, such a policy could use a ``Somewhere'' scope to mandate that at least one \controller{} deletes user data.
%
\Ces{} may wish to be more specific and ensure that a particular \controller{} obeys a policy,
in which case they should use the \emph{In} |name| scope.
%
This scope achieves greater precision, but is also tied to the source code; 
if the name of the \controller{} changes, the policy must change as well.
%

\subsection{Definitions}
\label{sec:definitions}

In the Freedit example from \S\ref{sec:overview}, the \ce{}'s final policy was:
\begin{lstlisting}[language=CNL]
For each "view" marked @@views@@:
  For each "database store" marked @@store@@:
    If "view" goes to "database store" then:
      There is a "date" marked @@time@@ where:
        "date" goes to "database store"
\end{lstlisting}
This policy contains five levels of nesting.
%
As policies get more complex, many levels of nesting can make policies inefficient and harder to understand.
%
To address this issue, \syslang{} allows \ces{} to create \emph{definitions}.
%
A definition defines a variable ahead of time which refers to all nodes in a given \controller{}'s PDG that meet a certain condition.
%
Observe that the Freedit policy does not enforce any obligations on a \lstinline[language=CNL]|"database store"| unless it stores a \lstinline[language=CNL]|"view"|.
%
Rather than iterate through \emph{all} database stores, 
a \writer{} can collect only the relevant database stores up front,
then write their policy in terms of those.
%
In this case, the Freedit \ce{} would create the following definition:
\begin{lstlisting}[language=CNL]
"view store" is each "store" marked @@db_store@@ where:
  There is a "view" marked @@views@@ where:
    "view" goes to "store"
\end{lstlisting}
and revise their policy to:
\begin{lstlisting}[language=CNL]
For each "view store":
  There is a "date" marked time where:
    "date" goes to "view store" 
\end{lstlisting}
This policy is also more efficient because it avoids the double |For each| loop of the original,
which may be expensive in an application that has many \lstinline[language=CNL]|"view"|s or \lstinline[language=CNL]|"database store"|s.
%

Since \sys{} constructs per-\controller{} PDGs,
\syslang{} policies, once compiled, are evaluated against one \controller{} at a time.
%
The scope of the policy dictates which \controller{}(s) must uphold the policy.
%
Since policy bodies only consider nodes in the current \controller{}'s PDG,
definitions are also, by default, \controller{}-specific.

However, a \ce{} may want to gather all nodes meeting a certain condition \emph{across \controller{}s}.
%
For instance, consider a policy that states that for each |sensitive| type that the application |store|s,
that type is also |delete|d.
%
It is unlikely that a single \controller{} would both store the sensitive data \emph{and} delete it.
%
Instead, the \ce{} could declare a definition to gather the sensitive types that are stored across the application,
then write a policy that states that some \controller{} must delete those types.
%
They would do so by appending ``anywhere in the application'' to their definition declaration, like so:

\begin{figure}[h]
\begin{lstlisting}[language=CNL]
Scope: Somewhere

Definitions:
"stored sensitive" is each "sensitive" type marked @@sensitive@@ where, 
anywhere in the application:
    There is "database store" marked @@store@@ where:
      "sensitive" goes to "database store"

Policy:
For each "stored sensitive":
    There is a "deleter" marked @@deletes@@ where:
      "stored sensitive" goes to "deleter"
\end{lstlisting}
\end{figure}

\subsection{Body}
\label{sec:body}

The policy body has three components: iterators, relations, and conjunctions/disjunctions of them.
%
\paragraph{Iterators: }
Iterators allow \ces{} to loop over a collection of nodes, reasoning about one node at a time.
%
\syslang{} provides two iterators: a |For each| loop or a |There is| loop.
%
To iterate over a defined variable \lstinline[language=CNL]|"x"|, 
a \ce{} would write \lstinline[language=CNL]|For each "x":| or \lstinline[language=CNL]|There is a "x" where:|.
%
To introduce a new variable,
a \ce{} would write \lstinline[language=CNL]|For each "x" marked @@m@@:| or \lstinline[language=CNL]|There is a "x" marked @@m@@ where:|.
%
A |For each| loop iterates over each object matching the condition and evaluates the body of the loop 
in the context of the current object \lstinline[language=CNL]|"x"|.
%
It succeeds if the body is true for all \lstinline[language=CNL]|"x"|.
%
If there are no \lstinline[language=CNL]|"x"|s, the body of the loop is vacuously true.
%
\syslang{} allows for vacuity for |For each| loops because it evaluates policies on a per-\controller{} basis,
and some \controller{}s may not have certain markers.
%
For example, take the (abbreviated) Freedit policy from \S\ref{sec:overview}:
\begin{lstlisting}[language=CNL]
Scope: Everywhere

Policy:
For each "view" marked @@views@@:
  For each "database store" marked @@store@@:
    [...]
\end{lstlisting}

Since the policy's scope is ``Everywhere'', it will be evaluated against every \controller{},
regardless of whether it handles view data.
%
It would be confusing for a \dev{} if their application failed the policy on a \controller{} 
that does not even contain \lstinline[language=CNL]|@@views@@|.
%
The \ce{} could check for vacuity by changing their scope to list only \controller{}s which they expect to handle 
\lstinline[language=CNL]|@@views@@| and \lstinline[language=CNL]|@@stores@@|,
but that would defeat \syslang's goal of being independent from source code.
%
A better way of checking for vacuity is to leverage the |There is| iterator, like so:
\begin{figure}
  \begin{lstlisting}[language=CNL]
    Scope: Everywhere
    
    Policy:
    There is a "view" marked @@views@@ where:
      There is a "database store" marked @@store@@ where:
        "view" goes to "database store"
    and
    For each "view" marked @@views@@:
      For each "database store" marked @@store@@:
        If "view" goes to "database store" then:
          There is a "date" marked @@time@@ where:
            "date" goes to "database store"
  \end{lstlisting}
  \label{f:vacuity}
\end{figure}
%
The |There is| iterator succeeds if the body is true for at least one \lstinline[language=CNL]|"x"|.
%
If there are no \lstinline[language=CNL]|"x"|s, the body of the loop is false.

Observe that the loop bodies reason about the iterator variables.
%
If instead, we eschewed iterators and wrote this policy as:
\begin{lstlisting}[language=CNL]
If a "view" marked @@views@@ goes to a "database store" marked @@store@@:
  There is a "date" marked @@time@@ where:
    "date" goes to a "database store" marked @@store@@
\end{lstlisting}
We would not enforce that \lstinline[language=CNL]|"view"| and \lstinline[language=CNL]|"date"| 
go to the \emph{same} \lstinline[language=CNL]|"database store"|.
%
Iterators allow for the correct version of the policy by allowing \ces{} to refer to the same object multiple times.

\begin{figure}
  \small
  \begin{tabular}{|p{5.5cm}|p{8cm}|}
      \hline
      \syslang{} Relation                                                       &  Obligation on PDG                   \\ \hline
      \lstinline[language=CNL]|"a" influences "b"|                              & \lstinline[language=CNL]|"a"| has transitive influence on \lstinline[language=CNL]|"b"|. \\
      \hline
      \lstinline[language=CNL]|"a" goes to "b"|                                 &  \lstinline[language=CNL]|"a"| has transitive data flow influence on \lstinline[language=CNL]|"b"|. \\
      \hline
      \lstinline[language=CNL]|"a" affects whether "b" happens|                 & \lstinline[language=CNL]|"a"| has transitive control flow influence on \lstinline[language=CNL]|"b"|. \\
      
      \hline
      \lstinline[language=CNL]|"a" goes to "b" only via "c"|                    & On every path from \lstinline[language=CNL]|"a"| to \lstinline[language=CNL]|"b"|,
                                                                                  \lstinline[language=CNL]|"a"| passes through \lstinline[language=CNL]|"c"|. \\
      \hline
      \lstinline[language=CNL]|"a" goes to "b"'s operation|                     & \lstinline[language=CNL]|"a"| goes to the call site associated with \lstinline[language=CNL]|"b"|
                                                                                 (e.g., if \lstinline[language=CNL]|"b"| is an argument to a call site, then \lstinline[language=CNL]|"a"| goes to any of that call site's arguments). \\
      \hline 
      \lstinline[language=CNL]|"a" is marked @@m@@|                             & \lstinline[language=CNL]|"a"| is marked \lstinline[language=CNL]|@@m@@|. \\
             
    \hline
  \end{tabular}
    \caption{\syslang's relations and the obligations they enforce on \sys's marked PDG.}
    \label{f:relations}
\end{figure}

\paragraph{Relations: }
%
Relations are between two objects: either two variables and a variable and a marker.
%
The full list of available relations are in Figure~\ref{f:relations}.
%
\syslang{} also supports the negation of each of these relations, e.g. \lstinline[language=CNL]|"a" does not go to "b"|.

\paragraph{Bullets: } 

In our examples thus far, we have used indentation to nest iterators.
%
However, such a design is error-prone--with just one accidental indentation, 
a \ce{} would write an entirely different policy than what they intended.
%
For instance, take the policies in Figure~\ref{f:indentation}, which differ only in indentation but have different meanings.
%
If there is no \lstinline[language=CNL]|"x"|, then Version 1 will fail.
%
Version 2, however, can still pass if \lstinline[language=CNL]|"y"| goes to \lstinline[language=CNL]|"z"|.

\begin{figure}[t]
    \begin{subfigure}[b]{\columnwidth}
  \begin{lstlisting}[language=CNL]
    There is a "x" where:
      "x" goes to "y"
      or
      "y" goes to "z"
  \end{lstlisting}
  \caption{Version 1 of the policy.}
  \end{subfigure}
  \begin{subfigure}[b]{\columnwidth}
  \begin{lstlisting}[language=CNL]
    There is a "x"
      "x" goes to "y"
    or
    "y" goes to "z"
    \end{lstlisting}
    \caption{Version 2 of the policy.}
    \end{subfigure}
    \caption{Two policies with identical syntax but different scopes. These policies are partial; we elide iterator declarations for brevity.}
    \label{f:indentation}
\end{figure}

\begin{figure}[t]
    \begin{subfigure}[b]{\columnwidth}
  \begin{lstlisting}[language=CNL]
    1. There is a "x" where:
      A. "x" goes to "y"
      or
      B. "y" goes to "z"
  \end{lstlisting}
  \caption{Version 1 of the policy.}
  \end{subfigure}
  \begin{subfigure}[b]{\columnwidth}
  \begin{lstlisting}[language=CNL]
    1. There is a "x" where:
      A. "x" goes to "y"
    or
    2. "y" goes to "z"
    \end{lstlisting}
    \caption{Version 2 of the policy.}
    \end{subfigure}
    \caption{The policies from Figure~\ref{f:indentation} with bullets to explicitly delineate each expressions's scope.}
    \label{f:bullets}
\end{figure}

Rather than allow a stray indent to change the meaning of the policy,
\syslang{} instead enforces that \ces{} explicitly specify the scope of each statement.
%
They do so with \emph{bullets}.
%
Figure~\ref{f:bullets} shows the policies from Figure~\ref{f:indentation} with \syslang{} bullets.
%
A bullet is followed by an iterator or a relation.
%
To introduce an additional bullet at the same level, \Ces{} use |and| or |or| operators.
%
\Ces{} are not permitted to mix operators (|and|s and |or|s) on the same bullet level,
since the operator precedence in such cases would be ambiguous.

\subsection{Types}
\syslang{} intentionally abstracts away details of the PDG to make policies easier to write.
%
Namely, \syslang{} does not support reasoning about particular source code entities (e.g., arguments, functions).
%
The one exception is types.
%
A typical data deletion policy will take the following form:
\begin{lstlisting}[language=CNL]
1. For each "sensitive" type marked @@user_data@@:
  A. There is a "source" that produces "sensitive" where:
    a. There is a "deleter" marked @@deletes@@ where:
        i) "source" goes to "deleter"
\end{lstlisting}
%
This policy enforces that for each user data type, some data of that type is deleted.
%
Consider how a \ce{} would write this policy without the |type| keyword.
%
They could write that \lstinline[language=CNL]|There is a "source" marked @@user_data@@| that is deleted,
but only checks that at least one \lstinline[language=CNL]|@@user_data@@| type is deleted, not that all of them are.
%
They could approximate the |type| keyword by giving each type a unique marker and using \lstinline[language=CNL]|There is|
iterators for each of them, but such a policy would be tedious to write and easy to get wrong.

\syslang{}'s full grammar is in \Cref{sec:grammar}.
