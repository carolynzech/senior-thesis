\chapter{Introduction}
\label{sec:intro}

\carolyn{todo: would like to get rid of the ``Chapter 1'' heading, but using chapter*
means the section numbers don't start over from 1 in each new chapter. come back and fix.}

Software providers that handle sensitive user data are subject to increasing regulatory scrutiny.
%
Regulations like the GDPR and CCPA give users unprecedented control over their data, including rights to access and deletion.
%
Enforcement agencies are levy large fines for violations.
%
In May of 2023 alone, Ireland fined Meta €1.2 billion for GDPR violations, 
while the United States fined Amazon \$25 million for failing to honor data deletion requests under the Children’s Online Privacy Protection Act.\cite{todo}
% https://www.nytimes.com/2023/05/22/business/meta-facebook-eu-privacy-fine.html?searchResultPosition=14
% https://www.nytimes.com/2023/12/20/technology/ftc-regulation-children-online-privacy.html?searchResultPosition=19
% https://www.nytimes.com/2023/05/31/technology/amazon-25-million-childrens-privacy.html?searchResultPosition=11
To ensure compliance, organizations must relate dense legal text to their low-level application source code.
%
Currently, software providers use manual audits to verify compliance, which are expensive, time-consuming,
and infrequent.~\cite{todo}

\sys{} is a static analyzer for Rust applications that \devs{} leverage to find such bugs in their programs before deployment.
%
\sys{} evaluates an application by generating its program dependence graph (PDG), 
where nodes are program entities (functions, arguments, etc.) and edges are control or data flow dependencies.
%
\sys{} evaluates the PDG against the policy and outputs whether the application is compliant.
%
We imagine that \devs{} run \sys{} continually throughout the development process, perhaps as part of a CI pipeline.

\section{The Problem}
Ideally, \writers{} could specify policies at a high level (e.g., ``all sensitive data is encrypted before it exits the application'').
%
Such a high-level specification is clear, concise, and portable across applications.
%
Prior to this work, however, \sys{} did not support this level of abstraction.
%
Instead, policies were Rust programs that executed low-level graph queries over the application's PDG.
%
This design limited the available pool of policy writers to those with an in-depth understanding of Rust programming and \sys{}'s PDG representation.
%
Since policies were programs, they had the same problems as the applications they evaluated: 
they were buggy, laborious to write, and comprehensible only to programmers familiar with the application.
%
It was crucial that we address this problem; the quality of \sys's analysis was moot if organizations could not use it easily (or even correctly).

We address this issue with \syslang{}, a DSL for \policies{} over PDGs.
%
Natural language offers the intuitive quality we sought, while the defined grammar allowed us to avoid the ambiguity of unconstrained English.
%
We pair \syslang{} with a compiler that translates policies to \sys{} graph queries.
%
\Devs{} can then invoke \sys{} on their application,
which will generate its PDG, evaluate it against the compiled \syslang{} policy,
and output whether the application is compliant.
%

\section{Goals}
\syslang{} must address a few challenges.
%
First, it must be accessible yet unambiguous.
%
\syslang{} should not require \writers{} to reason about low-level source code entities, such as particular functions or types.
%
Such a design would tie the policy to the application's implementation, effectively restricting policy writing to the application's \devs.
%
It would also make the policies brittle to source code changes and less portable across applications.
%
\syslang{} should instead be accessible to a broader range of technical stakeholders.
%
Ideally, a \ce--someone who knows how to program, but is not necessarily familiar with the particular implementation details--would write policies.
%
% Larger organizations have dedicated compliance teams that could write such policies.~\cite{todo}
% %
% In smaller organizations, the \ce{} could be a team's manager.\carolyn{this level of detail is probably not necessary}
% %
% Natural language is the most obvious intuitive structure for specifying these policies, 
% especially since privacy regulations are specified in natural language.\carolyn{I don't like this sentence.}
% %
% However, natural language is also ambiguous.
% %
% For example, the sentence ``|A| or |B| and |C|'' is vague; which operator should take higher precedence?
% %
% Programming languages have well-known constructs for handling such ambiguity, such as parenthesized expressions or operator precedence rules.~\cite{todo}
% %
Natural language is an obvious intuitive structure for specifying these policies.
%
However, unconstrained natural language can be ambiguous, which is unacceptable for a privacy policy.
%
\syslang{} must reject ambiguous policies while not becoming so onerous to use that it 
requires the same expertise and time investment as the existing Graph Query interface.

Second, \syslang{} should be expressive.
%
A trivial way of eliminating ambiguity would be to define a highly restricted natural language interface that allows for only a handful of policy formulations.
%
This structure, however, would not be expressive enough to meet the needs of complex applications, 
which must abide by a wide range of privacy requirements.~\cite{todo}
%
Instead, \syslang{} should be flexible enough to apply across policy and application domains.

Third, \syslang{} should be precise.
%
\sys{}'s PDG representation preserves dependencies at a high level of precision.
%
% It can reason about the dependencies of each distinct function argument or fields of a type.\carolyn{confusing sentence}
%
With the existing graph query interface, \writers{} can leverage this precision to make highly specific queries.
%
For instance, they can count the exact number of data flow edges between two particular nodes in the graph,
or query whether data flowed to a particular argument of a single call site.
%
However, \policies{} often do not require this level of precision.
%
If \syslang{} were to support it, it would not be accessible--it would be closely tied to the source code,
and would support such a wide breadth of query patterns that \writers{} would require expertise to select the correct ones.
%
However, if policies are not sufficiently precise, they are more likely to produce false positives or false negatives when run against application code.
%
% \carolyn{give an example of this? feels too weedy for an intro.}
%
% For certain policies, such precision is required for correctness.
% %
% For example, if only a particular field of a type is sensitive user data, then a \dev{} should apply a |sensitive| marker to that field only.
% \carolyn{hopefully the reasoning for the above is fairly intuitive? perhaps expand}
%
\syslang{} should support sufficiently precise queries to express common \policies{}, 
while not overwhelming the \writer{} with the full power of \sys{}'s PDG representation. 

Fourth, \syslang{} policies, once compiled, should be similarly efficient to graph queries originally written in Rust.
%
This goal is necessary so that \sys{} remains practical for CI use.

We evaluate \syslang{} against seven third-party Rust applications.
%
% We emulate the role of a \ce, formulating policies for the applications after reading documentation and using online demos.
%
\syslang{} can express a wide range of \policies{} for these applications.
%
These policies are sufficiently precise that they succeed on correct implementations and fail on buggy ones. 
%
They were often simpler than the equivalent Rust graph query policies that we had defined earlier, with no loss of correctness.

% This thesis makes three contributions:
% \begin{enumerate}[nosep]
%     \item The \syslang{} grammar for specifiying \policies.
%     \item The \syslang{} compiler to translate \policies{} to graph queries over a PDG.
%     \item An evaluation demonstrating that \syslang{} is expressive, precise, and efficient, with some limitations.
% \end{enumerate}

\emph{\sys{} and its Rust Policy API is the product of collaboration with Justus Adam, Livia Zhu, Sreshtaa Rajesh, 
Will Crichton, Shriram Krishnamurthi, and Malte Schwarzkopf.
My contribution specifically is the \syslang{} policy language and the compilation to Rust policies.}
