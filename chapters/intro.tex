\chapter{Introduction}
\label{sec:intro}

Privacy bugs in applications harm users (e.g., by leaking their sensitive data) and put software providers 
at risk for financial liability.
%
In May of 2023 alone, Ireland fined Meta €1.2 billion for GDPR violations, 
while the United States fined Amazon \$25 million for failing to honor data deletion requests 
under the Children’s Online Privacy Protection Act~\cite{meta-fine,amazon-fine}.
%
To ensure compliance, organizations must relate dense legal text to their low-level application source code.
%
Currently, organizations use manual audits to verify compliance, which are expensive, time-consuming,
and infrequent~\cite{CostContinuousCompliance2020,smithGDPRRacketWho}.
%

\sys{} is a static analyzer for Rust applications that \devs{} leverage to find such bugs in their programs before deployment.
%
\sys{} evaluates an application by generating its program dependence graph (PDG), 
where nodes are program entities (functions, arguments, etc.) and edges are control or data flow dependencies.
%
\sys{} evaluates the PDG against the policy and outputs whether the application is compliant.

\Devs{} can use \sys{} to receive immediate feedback on the impact of their changes by running 
\sys{} continually throughout the development process, perhaps as part of a CI pipeline.
%
We envision that \sys{} could replace (or at least reduce the frequency of) manual compliance audits.


\section{The Problem}
\sys{} policies are Rust programs that execute low-level graph queries over an application's PDG.
%
To write these programs, \devs{} must first connect high-level concepts 
(e.g., ``all sensitive data is encrypted'')
to low-level implementation details.
%
They must then understand how \sys{} represents these details with a PDG,
and finally translate their high level policy into a series of assertions about paths in this graph.
%
This design limits the available pool of policy writers to those with 
an in-depth understanding of Rust programming, application semantics, and \sys{}'s PDG representation.
%
Since policies are programs, they have the same problems as the applications they evaluate: 
they are buggy, laborious to write, and comprehensible only to programmers familiar with the application.

Ideally, policy writers would specify only the high-level concepts behind their policy,
which \sys{} would translate automatically into an application-specific query.
%
We realize this goal with \syslang{}, a DSL for \policies{} over PDGs.
%
\syslang{} is a Controlled Natural Language (CNL) interface; users write policies in natural language, 
but are restricted to a defined grammar ~\cite{cnl-def}.
%
Natural language offers the intuitive quality we seek, 
while the grammar allows us to avoid the ambiguity of unconstrained English.
%
We pair \syslang{} with a compiler that translates policies to \sys{} graph queries in Rust.
%
\Devs{} then invoke \sys{} on their application,
which generates its PDG, evaluates it against the compiled \syslang{} policy,
and outputs whether the application is compliant.
%

\section{Goals}
The design of \syslang{} must address four challenges.

First, it must be accessible yet unambiguous.
%
\syslang{} should avoid requiring policy writers to reason about low-level source code entities, 
such as particular functions or arguments.
%
Such a design would tie the policy to the application's implementation, 
effectively restricting policy writing to the application's \devs.
%
It would also make the policies more difficult to write and brittle to source code changes.
%
\syslang{} should instead be accessible to a range of technical stakeholders.
%
Ideally, a \ce--someone who knows how to program, 
but who is not necessarily familiar with the particular implementation details--would write policies in a code-agnostic way.
%

Natural language provides an intuitive structure for specifying these policies.
%
However, unconstrained natural language can be ambiguous.
%
For example, the sentence ``|A| or |B| and |C|'' is vague; which operator should take higher precedence?
%
While such ambiguity may be acceptable (or even desirable) in legal text,
it is unacceptable for a policy that is checkable by an automated tool.
%
\syslang{} must reject ambiguous policies without becoming so onerous to use that it 
requires the same expertise and time investment as writing a graph query program.

Second, \syslang{} should be expressive.
%
A trivial way of eliminating ambiguity would be to define a highly restricted natural language interface that allows for only a handful of policy formulations.
%
This structure, however, would be insufficiently expressive to meet the needs of complex applications, 
which must abide by a wide range of privacy requirements.
%
Instead, \syslang{} should be flexible enough to apply across policy and application domains.

Third, \syslang{} should support sufficiently precise queries to express common \policies{}, 
without overwhelming the \writer{} with the specifics of \sys{}'s PDG representation. 
%
\sys{}'s PDG representation represents program structure with a specific level of precision.
%
With the existing graph query interface, \writers{} can leverage this precision to make highly specific queries
(e.g., count the exact number edges between two nodes in a graph). 
%
However, \policies{} often do not require this level of precision.
%
If \syslang{} were to require reasoning about the PDG specifics, 
it would not be accessible--\ces{} would require expertise
to choose from a wide array of query patterns over a complex, unfamiliar data structure.
%
However, if policies are insufficiently precise, they are more likely to produce false positives or false negatives when run against application code.

Fourth, \syslang{} policies, once compiled, should be similarly efficient to graph queries written in Rust.
%
This goal is necessary so that \sys{} runs at interactive timescales when checking \syslang{} policies.

We evaluate \syslang{} against seven third-party Rust applications.
%
We find that \syslang{} can express a wide range of \policies{} for these applications,
that these policies correctly identify bugs,
and that they incur acceptable run time overhead (\textasciitilde{}0-12\%) compared to hand-optimized Rust graph queries.

\bigskip{}
\emph{\sys{} and its Graph Query API is the product of collaboration with Justus Adam, Livia Zhu, Sreshtaa Rajesh, 
Will Crichton, Shriram Krishnamurthi, and Malte Schwarzkopf.
My contribution specifically is the \syslang{} policy language and the compilation to Rust policies.}
