\chapter{Overview}
\label{sec:overview}

\Ces{} and \devs{} collaborate to apply \sys{} to their applications and catch privacy bugs before deployment.
%
We demonstrate \syslang's role in this process through an example.

Freedit is a text-based social media application, similar to Reddit or Twitter~\cite{freedit}.
%
The application stores a user's viewing history, but deletes it after three days~\cite{freedit-pageviews}.
%
To verify that the source code upholds the policy, 
a \ce{} must first ensure that any viewing data is always stored alongside the current date.
%
Otherwise, the application would have no way to check later whether the data is expired.
%
To encode this policy in \syslang{}, the \ce{} first considers which operations or data are relevant to the policy.
%
They determine that the relevant concepts for this policy are \emph{(i)} the viewing data,
\emph{(ii)} any database store operations, and \emph{(iii)} the current date.
%
They define \lstinline[language=CNL]|@@views@@|, 
\lstinline[language=CNL]|@@store@@|, 
and \lstinline[language=CNL]|@@time@@| markers to represent each of these ideas.
%
The \ce{} must have some technical background to establish these concepts--they must, for example, understand what a database is,
and the kind of data that it would store.
%
However, they do not need to understand how the source code implements this functionality,
or the details of the database client library that the app uses.
%

Next, the \ce{} writes the policy in terms of these markers.
%
They know that a \dev{} will eventually apply their markers to one or more source code entities.
%
Their policy must therefore reason about these marked object(s).
%
To do so, the \ce{} introduces \emph{variables} into their policy.
%
A variable represents a single source code entity with a given marker.
%
In this example, the \ce{} introduces their variables as follows:
\begin{lstlisting}[language=CNL]
For each "view" marked @@views@@:
  For each "database store" marked @@store@@:
\end{lstlisting}
%
Here, and in the following text, variables are in purple and markers are in teal.

The \ce{} must now establish that a given \lstinline[language=CNL]|"database store"| actually stores this \lstinline[language=CNL]|"view"|--
if it stores some other, unrelated data, the policy does not apply.
%
To do so, they introduce the first \emph{predicate} into their policy:
\begin{lstlisting}[language=CNL]
For each "view" marked @@views@@:
  For each "database store" marked @@store@@:
    /*If*/ "view" /*goes to*/ "database store" /*then:*/
\end{lstlisting}
%
Predicates are expressions that relate two objects to each other--either two variables, or a variable and a marker.

Note that the \ce{} need not specify \emph{how} or \emph{when} the \lstinline[language=CNL]|"view"| reaches the \lstinline[language=CNL]|"database store"|.
%
The \lstinline[language=CNL]|"view"| could undergo any number of data transformations first--e.g., it could be inserted as a field of a type containing other user data,
and that type is then stored.
%
The \ce{} does not worry about these implementation details; they must only understand that the \lstinline[language=CNL]|"view"| reaches the \lstinline[language=CNL]|"database store"|
in some form, at some point.
%

The \ce{} then defines the consequent of the conditional:
\begin{lstlisting}[language=CNL]
For each "view" marked @@views@@:
  For each "database store" marked @@store@@:
    If "view" goes to "database store" then:
      /*There is a*/ "date" /*marked*/ @@time@@ /*where:*/
        "date" /*goes to*/ "database store"
\end{lstlisting}
%
The consequent ensures that if the view is stored, that some date is also stored.
%
% Since both the antecedent and the consequent reason about the same \lstinline[language=CNL]|"database store"| variable,
% the \ce{} ensures that the date and view are stored \emph{together}.
%
% Without variables, the \ce{} could only write that the current date goes to some entity marked \lstinline[language=CNL]|@@store@@|,
% but this could be a completely different store unrelated to \lstinline[language=CNL]|"view"|.

Finally, the \ce{} establishes the \emph{scope} of their policy,
%
For instance, a GDPR data deletion policy would assert that somewhere in the application, 
all of a user's data is deleted.
%
In this case, the \ce{} determines that the entire application should uphold this policy, so they select the ``Everywhere'' scope,
producing the final policy:

\begin{lstlisting}[language=CNL]
/*Scope: Everywhere*/

/*Policy:*/
For each "view" marked @@views@@:
  For each "database store" marked @@store@@:
    If "view" goes to "database store" then:
      There is a "date" marked @@time@@ where:
        "date" goes to "database store"
\end{lstlisting}

The \ce{} then sends this policy to a \dev{}, 
who leverages their implementation knowledge to apply the markers to the appropriate source code entities.
%
Here, they might attach the \lstinline[language=CNL]|@@views@@| marker on all function arguments of type |PageView|,
the \lstinline[language=CNL]|@@store@@| marker on an argument to a database library insertion function,
and the \lstinline[language=CNL]|@@time@@| marker on the return value of a function that gets the current date.
%
The \dev{} runs \syslang{}'s compiler, 
which outputs a Rust program that contains an equivalent graph query over a marked PDG 
and boilerplate code to invoke \sys{}.
%
\sys{} generates the application's marked PDG, 
evaluates it against the policy,
and outputs whether the policy passed or failed.
%
