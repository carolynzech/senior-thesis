\section{Related Work}

A \policylang{} should be \textbf{accessible}, \textbf{expressive}, and \textbf{precise}.
%
Existing systems meet at most two of these goals.
%

Program-based specifications allow developers to write precise and expressive policies.
%
In Resin ~\cite{resin}, developers specify dynamic dataflow assertions in their application code.
%
Ponder ~\cite{ponder} also encodes policies through code; developers specify assertions that must hold or code that should run if a given predicate is satisfied.
%
Other interfaces, such as IFC ~\cite{jif} or regular expression-based rules ~\cite{hotnets}, 
allow for precise reasoning over graphs without writing code, but still require a strong technical background.

Legalese ~~\cite{legalese} is a \policylang{} that, like \syslang, is structured around allowing or disallowing information flows over a data dependency graph.
%
Legalese can only express whether one node does or does not flow to another--it cannot express control flow or the order of nodes in a dataflow path.
%
% It also applies policies to every path in the graph, while \syslang{} supports policy scope.
%
While Legalese's \todo{prose-based structure} attempts to make it more accessible than program-based specifications, 
user surveys have found that non-programmers struggle to define or interpret Legalese policies ~\cite{legalese, privguard}. \todo{could be good to say why}

Riverbed ~\cite{riverbed}, RuleKeeper ~\cite{rulekeeper}, and Logical English ~\cite{logical-english} offer more accessible policy specifications.
%
However, they are very limited to particular domains--Riverbed encodes four binary allow/disallow attributes, while RuleKeeper supports a subset of GDPR provisions.
%
Unlike \syslang, which is not dependent on a particular programming language as a compilation target, Logical English is merely syntactic sugar for Prolog.

\syslang{} is a \policylang{} that leverages \sys{}'s marked PDG abstraction to maintain acceptable precision compared to most program-based policies,
while still providing an intuitive and expressive \todo{interface}.